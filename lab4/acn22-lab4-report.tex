\documentclass[a4paper,11pt]{article}

\usepackage[T1]{fontenc}
\usepackage[utf8]{inputenc}

\usepackage{multirow}
\usepackage{booktabs}
\usepackage{graphicx}
\usepackage{setspace}
\usepackage[skip=6pt plus1pt, indent=0pt]{parskip}

\usepackage{float}
\usepackage{fancyhdr}

\usepackage{tcolorbox}
\usepackage{hyperref}
\hypersetup{
    colorlinks=true,
    linkcolor=blue,
    filecolor=magenta,
    urlcolor=blue
}

\usepackage[margin=1in]{geometry}

\newcommand{\incode}[1]{
\begin{tcolorbox}[colback=blue!5!white, boxrule=0mm, sharp corners]
\texttt{#1}
\end{tcolorbox}
}

\newcommand{\note}[1]{\textit{\textcolor{gray}{#1}}}

\pagestyle{fancy}
\fancyhf{}
\lhead{Advanced Computer Networks 2022}
\rhead{Lin Wang, George Karlos, Florian Gerlinghoff}
\cfoot{\thepage}

\begin{document}


\thispagestyle{empty}

\begin{tabular}{@{}p{15.5cm}}
{\bf Advanced Computer Networks 2022} \\
Vrije Universiteit Amsterdam  \\ Lin Wang, George Karlos, Florian Gerlinghoff\\
\hline
\\
\end{tabular}

\vspace*{0.3cm}

{\LARGE \bf Lab4: Who Is Watching My Video? (Report)}

\vspace*{0.3cm}

%============== Please do not change anything above ==============%

% Please modify this part with your group information
\begin{tcolorbox}[sharp corners, colback=blue!5!white]
\begin{tabular}{@{}ll}
\textbf{Group number:} & 28 \\
\textbf{Group members:} & David Freina, Jonas Wagner \\
\textbf{Slip days used:} & 0 \\
\end{tabular}
\end{tcolorbox}

\vspace{0.4cm}

% Please do not remove any of the section headings below

\section{Implementing an IPv4 Router in P4}
\label{sec:implementing-router}
% Start your writing here, feel free to include subsections to structure your report if needed
% Please remove the note below in your submission
%\note{Please explain the high-level idea, as well as important details in your implementation.}

First we start by implementing the required headers with their respective bit fields according to the lecture slides as well as the P4 documentation.
Apart from that we also add a struct to combine the ipv4 and ethernet header to comply with the given parser/control-block signatures.
The parser \textit{MyParser} is implemented similar to the parser presented on the lecture slides though we omit the TCP and UDP parsing.
Our next step is to complete the ingress processing by adding to actions: \textit{drop} and \textit{ipv4\_forward}.
To tell the router when it needs to take which action we define a table which matches based on the longest-prefix match for the IPv4 destination address.
If the address matches the packet gets forwarded to the specified port, the TTL of the packet is decreased by 1 and the source and destination mac addresses are updated accordingly.
Last but not least we implement the \textit{MyComputeChecksum} as well as \textit{MyDeparser} analogous to the lecture slides.\\
In addition to the router implementation we update the s1- to s3-runtime.json to provide correct static forwarding tables with the necessary entries for the switches.
Testing our implementation by using mininets \textit{pingall} command works correctly.


\section{Intercepting RTP Video Streaming with P4}

% Start your writing here, feel free to include subsections to structure your report if needed
% Please remove the note below in your submission
%\note{Please explain the high-level idea, as well as important details in your implementation.}

\subsection{Streaming h1-h7}

In order to implement the streaming for h1 to h7 we re-use our router implementation from \autoref{sec:implementing-router}.
We then fill in the details needed for the correct routing in the s1- to s7-runtime.json by adding table, match and action entries.

\subsection{Intercepting the stream}

We tried intercepting the stream by using a multicast group defined in s2-runtime.json (l. 55-69) which allowed us to forward all packets to two different ports.
This implementation worked in the sense that we were able to see all the packets from the RTP stream on h3 as well as h7.
Furthermore, we were able to start the stream on h7 but unable to start it on h3.

Due to this limitation we decided to try and clone the packets which arrive on s2 and forward the cloned packets to s3.
To allow us to do this we have to edit the topology slightly by adding the parameter \textit{cli\_input} to s2 (topology.json: l. 34).
This parameter allows us to load the commands stored in the file \textit{s2-commands.txt} which enables us to clone the packets on a specific port with a specific group id.
We then proceed by checking in the ingress pipeline if the packet has certain destination and source IP addresses (namely 10.0.7.7 and 10.0.1.1).
If those addresses match we clone the packets with the \textit{clone\_packet} method from ingress to egress.

In the egress pipeline we check if the packet was cloned and than proceed by rewriting the headers for those packets (methods change\_h1\_to\_h7\_addr and change\_h7\_to\_h1\_addr).
Depending on which direction the packets were going we rewrite the source and destination IP and MAC addresses as well as the egress port accordingly.

With this new implementation in place we are again able to see the RTP packets arriving at h3 and h7.
But similar to our first implementation we are again unable to playback the stream on h3 while the playback on h7 still works.
However, we than add the checksum recalculation for the extracted UDP header and the streaming now works on both h3 and h7 simultaneously.
% We are unsure if the multicast implementation would have worked with the correct checksum calculation.

% \subsubsection{Debugging steps to enable playback on h3}

% At first we were thinking that our multicast implementation cannot work because a multicast stream would need a source IP address of 224.0.0.0 through 239.255.255.255.
% As mentioned before we tried to get around this issue by cloning the packets rather than forwarding with the multicast which also did not work.

% Furthermore, we were able to see some ICMP packets from h7 to h1 which we try to clone as well.
% For the cloning to work correctly we changed the source IP address from 10.0.7.7 to 10.0.3.3.
% After this did not work we realized that this ICMP message embeds the last IP header which again had the wrong address set.
% We decided to change this address as well which then lead to a checksum mismatch error which we observed in WireShark.

% This is the current state of the implementation and no further debugging steps were taken.


\end{document}
